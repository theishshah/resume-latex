%%%%%%%%%%%%%%%%%%%%%%%%%%%%%%%%%%%%%%%%%
% Medium Length Professional CV
% LaTeX Template
% Version 2.0 (8/5/13)
%
% This template has been downloaded from:
% http://www.LaTeXTemplates.com
%
% Original author:
% Trey Hunner (http://www.treyhunner.com/)
%
% Important note:
% This template requires the resume.cls file to be in the same directory as the
% .tex file. The resume.cls file provides the resume style used for structuring the
% document.
%
%%%%%%%%%%%%%%%%%%%%%%%%%%%%%%%%%%%%%%%%%

%----------------------------------------------------------------------------------------
%	PACKAGES AND OTHER DOCUMENT CONFIGURATIONS
%----------------------------------------------------------------------------------------

\documentclass{resume} % Use the custom resume.cls style
    
    \title{ish-shah-resume}
    
    \usepackage[left=0.75in,top=0.6in,right=0.75in,bottom=0.6in]{geometry} % Document margins
    
    \usepackage{enumitem}
    \usepackage{multicol}
    \usepackage{hyperref}
    \usepackage[T1]{fontenc}
    
    \newcommand{\tab}[1]{\hspace{.2667\textwidth}\rlap{#1}}
    \newcommand{\itab}[1]{\hspace{0em}\rlap{#1}}
    
    \name{Ish Shah}
    
    \nameskip\break
    \address{(303)~802~7919 \\ 
    \href{mailto:ish@ishshah.me}{ish@ishshah.me} \\ 
    \href{http://github.com/theishshah}{GitHub: theishshah} \\ 
    \href{http://linkedin.com/in/theishshah}{LinkedIn: theishshah}}
    
    \begin{document}
    
    \begin{rSection}{Experience}
    
    \begin{rSubsection}{Ampaire}{September 2018 - December 2018}{Software Engineering Intern}{Los Angeles, CA}
    \item Implemented system to read serial data from flight instruments in real time
    \item Designed dashboards for monitoring sensor status and instrument data in flight
    \item Built custom flight instrumentation hardware, integrating GPS, IMU, and Air Data probes
    \item Integrated DAQ hardware with aircraft auxiliary power systems
    \end{rSubsection}
    
    \begin{rSubsection}{CoreOS}{May 2018 - September 2018}{Software Engineering Intern}{San Francisco, CA}
    \item Built Kubernetes operator for the CoreOS \verb+dex+ OpenID Connect provider
    \item Refactored code generation package to be more modular, increasing developer productivity 
    \item Implemented TravisCI pipeline to build and deploy operators
    \item Added support in the \verb+operator-sdk+ for higher configurability and better debugging of operators
    \end{rSubsection}
    
    \begin{rSubsection}{Ocient}{January 2018 - May 2018}{Software Engineering Intern}{Chicago, IL}
    \item Implemented system to mock high performance data center hardware
    \item Used Intel \verb+spdk+ to determine NVME drive queue depths and efficiently queue jobs 
    \item Built an integration testing pipeline for complex database operators    
    \end{rSubsection}
    
    \begin{rSubsection}{Bazaarvoice}{May 2017 - August 2017}{Software Engineering Intern}{Austin, TX}
    \item Worked on service gateway and authentication team
    \item Designed and implemented user activity logging service using Dropwizard
    \item Deployed Elasticsearch instance to store and query text records
    \end{rSubsection}
    
    \begin{rSubsection}{Coordinated Science Laboratory}{February 2017 - December 2017}{Undergraduate Researcher, Robotics}{Urbana, IL}
    \item Member of Intelligent Robotics Laboratory in the UAV Autopilot group
    \item Used PX4 to run custom navigation algorithms on drones
    \item Assembled and debugged low power embedded hardware
    \item Worked on implementing various parallel computation strategies
    \end{rSubsection}
    
    \end{rSection}
    
    \begin{rSection}{Education}
    
    \begin{rSubsection}{University of Illinois}{August 2016 - December 2017}{Computer Science}{Urbana-Champaign, IL}
    \item Chair of the GNU/Linux User Group. Taught students Linux fundamentals and OS tools
    \item Organizer for Hacker Experience at HackIllinois and for Speaker Content at Reflections|Projections
    \item Course Staff for CS196. Gave lectures on introductory computer science material
    \end{rSubsection}
    
    \end{rSection}
    
    \begin{rSection}{Proficiencies}
    
    \begin{tabular}{ @{} >{\bfseries}l @{\hspace{6ex}} l }
    Programming Languages &  Golang, Python, C/C++, Java, Ruby, Bash \\
    Developer Technologies & Docker, Kubernetes, Git, Elasticsearch, PostgreSQL, Flask, \LaTeX \\
    \end{tabular}
    
    \end{rSection}
    
    %\begin{rSection}{Relevant Courses}
    %\itab{\textbf{Core Courses}} \tab{}  \tab{\textbf{Other Courses}}
    %\\ \itab{Fluid Mechanics \& its applications } \tab{}  %\tab{Computational Methods in Engineering}
    %\\ \itab{Thermodynamics} \tab{}  \tab{Fundamental of Computing} 
    %\\ \itab{Heat Transfer \& its applications} \tab{}  \tab{Probability and Statistics} 
    %\\ \itab{Mass Transfer \& its applications} \tab{} \tab{Calculus \& Linear Algebra}
    %\\ \itab{Transport Phenomena (ongoing)} \tab{} \tab{Introduction to Mechanics}
    %\end{rSection}
    
    \end{document}
    